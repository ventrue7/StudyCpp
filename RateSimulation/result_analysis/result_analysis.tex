\documentclass[12pt]{article}
\usepackage{amsmath}

\title{LIBOR Market Model Result Analysis}
\begin{document}
\maketitle


\section{Validate Volatility Assumption}

After we generated LIBOR rates, we need to validate if the standard deviation of rates distribution still match the given volatility assumption.

\subsection{Theory}

\subsubsection{Volatility Parametrization}
Volatility formula, 
\begin{equation}
    \sigma_i(t) = [a(T_{i-1}-t)+d]e^{-b(T_{i-1}-t)}+d
\end{equation}
\\
This form allows a humped shape in the graph of the instantaneous volatility

\subsubsection{LIBOR Market Model Simulation}
LIBOR Market Model,
\begin{equation}
    dL_n(t_i)=\mu_n(t_i) L_n(t_i)dt+\sigma_n(t_i)L_n(t_i)dW_n(t_i)
\end{equation}
\\
Discrete form,
\begin{equation}
    L_n(t_{i+1})=L_n(t_i)+\mu_n(t_i) L_n(t)(t_{i+1}-t_i)+\sigma_n(t_i) L_n(t_i) \sqrt{t_{i+1}-t_i}Z^i_n
\end{equation}
\\
Integrate this equation, 
\begin{equation}
    L_n(t_{i+1})=L_n(t_i)
    \exp{\{(\mu_n(t_i)+\frac{1}{2}\sigma^2_n(t_i))(t_{i+1}-t_i)+\sigma_n(t_i) \sqrt{t_{i+1}-t_i}Z^i_n\}} 
\end{equation}


\subsubsection{The distribution of LIBOR rates}
The rates from LIBOR Market Model follow a lognormal distrinbution,
$
    log(L_n(t_{i+1})) \sim N\left( log(L_n(t_{i}))+(\mu_n(t_i)+\frac{1}{2}\sigma^2_n(t_i))(t_{i+1}-t_i),
    \sigma^2_n(t_i) (t_{i+1}-t_i)        \right)
$.
We need to validate if the variance of $log(L_n(t_{i+1}))$ distribution matches $\sigma^2_n(t_i) (t_{i+1}-t_i)$.

\subsection{An Example}
We would like to validate the volatility assumption of 3 Month length LIBOR rate starting from 9 Month to 12 Month with forward starting at 9 Month,
we will write it as $L_{0.75}(0.75)$ for short.

\subsubsection{Calculate the Volatility}
We used $L_{0.75}(0.75)$ to simulate $L_{0.75}(0.5)$, and $L_{0.75}(0.5)$ is from simulate through $L_{0.75}(0.25)$. Finally, we have 
\begin{equation}
    L_n(t_{k+1})\sim N\left( log(L_n(t_{0}))+\sum^k_{i=1}(\mu_n(t_i)+\frac{1}{2}\sigma^2_n(t_i))(t_{i+1}-t_i),
    \sum^k_{i=1} \sigma^2_n(t_i) (t_{i+1}-t_i)        \right)
\end{equation}
\\
In the volatility formula, $t$ is 0.25, 0.5, 0.75, $T_i$ is 1, 1.25, 1.5. In this example, we use $a = 0.19, b = 0.97, c = 0.08, d = 0.01$, we obtained  
$\sum^k_{i=1}\sigma^2_n(t_i) (t_{i+1}-t_i) = 0.017712$.

\subsubsection{Calculate the Variance}
We can take the log of the rates and apply sample variance formula to get the variance.

\subsubsection{Chi-Square Test for the Variance}
We can use a Chi-Square Test to validate if $\sigma^2 = \sigma^2_0$.
\\
$H_0:$
$
\sigma^2 = \sigma^2_0
$
\\
$H_a:$
$
\sigma^2 \neq \sigma^2_0
$
\\
Test Statistic:
$
T = \frac{N-1}{S^2/\sigma_0^2}
$
\\
where, $N$ is the smaple size, $S$ is the sample standard, $\sigma_0$ is the target standard deviation.
\\
Critical region: Reject $H_0$ if $T < \chi^2_{\alpha/2,N-1}$ or $T > \chi^2_{1-\alpha/2,N-1}$

\subsubsection{Result}
Absoulate Error: $ABS(\hat{\sigma^2} -\sigma^2_0)$.
\\
Percentage Error: $\frac{\hat{\sigma^2}-\sigma^2_0}{\sigma^2_0}$.
\\
\\
$N = 100$, 
Calculated Variance: 0.0147492, Target Variance: 0.017712, ABS Error: 0.00296284, Percentage Error: -0.167278
\\
\\
$N = 1000$, 
Calculated Variance: 0.0179749, Target Variance: 0.017712, ABS Error: 0.000262848, Percentage Error: 0.0148401
\\
\\
$N = 10000$,
Calculated Variance: 0.0175451, Target Variance: 0.017712, ABS Error: 0.00016692, Percentage Error: -0.00942413
\\
\\
$N = 100000$,
Calculated Variance: 0.0178083, Target Variance: 0.017712, ABS Error: 9.63033e-05, Percentage Error: 0.00543717
\\
\\
$N = 1000000$,
Calculated Variance: 0.0177014, Target Variance: 0.017712, ABS Error: 1.06467e-05, Percentage Error: -0.000601098
\\
\\
$N = 100$, Test Statistic: 90.3411, Degree of Freedom: 99, 
$\chi^2_{0.025,99} = 73.361$, $\chi^2_{0.975,99} = 128.422$
The test statistic value of 90.3411 is within the range of $\left(\chi^2_{\alpha/2,N-1}, \chi^2_{1-\alpha/2,N-1}\right)$,
so we cannot reject the null hypothesis.

\bibliographystyle{abbrv}
\bibliography{main}

\end{document}