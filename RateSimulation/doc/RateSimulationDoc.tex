\documentclass[12pt]{article}
\usepackage{amsmath}

\title{Simulate Libor Market Model}
\begin{document}
\maketitle


\section{Model}


Libor Market Model,
\begin{equation}
    dL_n(t_i)=\mu_n L_n(t_i)dt+\sigma_n(t_i)L_n(t_i)dW_n(t_i)
\end{equation}
\\
Discrete form:
\begin{equation}
    L_n(t_i+1)=L_n(t_i)+\mu_n L_n(t)(t_{i+1}-t_i)+\sigma_n(t_i) L_n(t_i) \sqrt{t_{i+1}-t_i}Z^i_n
\end{equation}
\\
where, $p$ is the numeraire.
\begin{equation}
    \mu_n =\left\{
\begin{aligned}
& -\sum_{k=n+1}^p \frac{\tau_k L_n(t)}{1+\tau_k L_n(t)}\rho_{kn} dt, n<p\\
& 0, n = p\\
& \sum_{k=p+1}^j \frac{\tau_k L_n(t)}{1+\tau_k L_n(t)}\rho_{kn} dt, n>p\\
\end{aligned}
\right.
\end{equation}

\section{Simulation Process}

Use time 0 rates to solve $L_0(t_i)$.
\\
Loop all the time step
\\
Loop Libor rates
\\
Calculate $\mu_n$ based on the comparsion of numeraire and $n$.
\\
Calculate $L_n(t_i+1)$
\\
End Libor rates
\\
End time step loop
\bibliographystyle{abbrv}
\bibliography{main}

\end{document}